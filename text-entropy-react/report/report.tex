\documentclass[12pt, a4paper]{article}
\usepackage[a4paper, includeheadfoot, mag=1000, left=2cm, right=1.5cm, top=1.5cm, bottom=1.5cm, headsep=0.8cm, footskip=0.8cm]{geometry}
% Fonts
\usepackage{fontspec, unicode-math}
\setmainfont[Ligatures=TeX]{CMU Serif}
\setmonofont{CMU Typewriter Text}
\usepackage[english, russian]{babel}
% Indent first paragraph
\usepackage{indentfirst}
\setlength{\parskip}{5pt}
% Diagrams
\usepackage{graphicx}
\usepackage{float}
% Page headings
\usepackage{fancyhdr}
\pagestyle{fancy}
\renewcommand{\headrulewidth}{0pt}
\setlength{\headheight}{16pt}
%\newfontfamily\namefont[Scale=1.2]{Gloria Hallelujah}
\fancyhead{}

\usepackage{listings}
\begin{document}

% Title page
\begin{titlepage}
\begin{center}

\textsc{ФГАОУ ВО «Санкт-Петербургский национальный исследовательский университет информационных технологий, механики и оптики»\\[4mm]
Факультет программной инженерии и компьютерной техники}
\vfill
\textbf{Лабораторная работа №1\\[4mm]
Прикладная математика\\[16mm]
}
Саржевский Иван
\\[2mm]Группа P3302
\vfill
Санкт-Петербург\\[2mm]
2019 г.

\end{center}
\end{titlepage}

\section*{Цель работы}
Получить практичские навыки решения задач на количественное измерение
информационного объема текстовой информации

\section*{Задание}
\begin{enumerate}
  \item Реализовать процедуру вычисления энтропии для текстового файла. В
    процедуре необходимо подсчитывать частоты появления символов (прописные и
    заглавные буквы не отличаются, знаки препинания рассматриваются как один
    символ, пробел является самостоятельным символом), которые можно использовать
    как оценки вероятностей появления символов. Затем вычислить величину энтропии.
    Точность вычисления - 4 знака после запятой. Обязательно предусмотреть
    возможность ввода имени файла, для которого будет вычисляться энтропия.
  \item Проверить запрограммированную процедуру на нескольких файлах и заполнить
    таблицу 1.1 вычисленными значениями энтропии
  \item Вычислить значение энтропии для тех же файлов, но с использованием
    частот вхождений пар символов и заполнить таблицу 1.2.
  \item Проанализировать полученные результаты
\end{enumerate}

\section*{Реализация процедуры}
\lstinputlisting[basicstyle=\ttfamily\scriptsize]{code.js}

\section*{Результаты}\
\subsection*{Первое задание}
Файл: Преступление и Наказание, 30000 символов

Энтропия: 2.9209

\begin{tabular}{ | l | l | l | }

\hline

Символ & Вероятность & Энтропия \\

\hline

0 & 0.0001 & 8.9111 \\

1 & 0.0002 & 8.6880 \\

2 & 0.0002 & 8.6880 \\

4 & 0.0001 & 9.1988 \\

5 & 0.0001 & 9.1988 \\

6 & 0.0001 & 9.1988 \\

7 & 0.0000 & 10.2974 \\

8 & 0.0002 & 8.3515 \\

9 & 0.0001 & 9.6043 \\

t & 0.0718 & 2.6345 \\

h & 0.0482 & 3.0327 \\

e & 0.0953 & 2.3504 \\

& 0.1637 & 1.8094 \\

p & 0.0134 & 4.3160 \\

r & 0.0436 & 3.1327 \\

o & 0.0628 & 2.7680 \\

j & 0.0010 & 6.8962 \\

c & 0.0180 & 4.0189 \\

g & 0.0173 & 4.0591 \\

u & 0.0221 & 3.8128 \\

n & 0.0566 & 2.8715 \\

b & 0.0127 & 4.3652 \\

k & 0.0088 & 4.7329 \\

f & 0.0166 & 4.0989 \\

i & 0.0525 & 2.9475 \\

m & 0.0210 & 3.8645 \\

a & 0.0612 & 2.7941 \\

d & 0.0324 & 3.4294 \\

s & 0.0511 & 2.9736 \\

. & 0.0549 & 2.9029 \\

y & 0.0154 & 4.1749 \\

v & 0.0077 & 4.8637 \\

w & 0.0187 & 3.9785 \\

l & 0.0300 & 3.5051 \\

x & 0.0012 & 6.7421 \\

q & 0.0008 & 7.1619 \\

z & 0.0003 & 8.2180 \\

\hline
\end{tabular}

\subsection*{Второе задание}
\begin{tabular}{| l | l | l | l |}
  \hline
  Файл & crime\_and\_punishment\_30 & the\_master\_and\_margarita\_30 \\
  \hline
  Энтропия H(X) & 2.9209 & 2.9337 \\
  \hline
  Энтропия H*(X) & 0.4961 & 0.4800 \\
  \hline
  Файл & war\_and\_peace\_30 \\
  \hline
  Энтропия H(X) & 2.9282 \\
  \hline
  Энтропия H*(X) & 0.4989 \\
  \hline
\end{tabular}

\section*{Выводы}
Протестировав три различных файла, состоящих из 30000 символов установил, что
значение энтропии (как с условием встречи одиночного символа, так и пары)
практически одинаково для всех трех файлов, что объясняется осмысленностью
текста.
\end{document}

